\title{Vorbis audio coding}
\author{Vicente González Ruiz}
\maketitle
\tableofcontents

\section{Steps}
\begin{enumerate}
\def\labelenumi{\arabic{enumi}.}
\tightlist
\item
  Overlaped subband analysis using
  \href{http://en.wikipedia.org/wiki/Modified_discrete_cosine_transform}{MDCT}.
\item
  Perceptual quantization.
\item
  Entropy coding.
\end{enumerate}

\section{Overlaped processing}
\begin{verbatim}
0              N-1            2N-1            3N-1
+---------------+---------------+---------------+ s[n]
<--------Transform Step--------->
                <---------Transform Step-------->
\end{verbatim}

\begin{itemize}
\item
  Each transform step inputs \(2N\) samples and outputs \(N\) MDCT
  coeficients.
\item
  \(N\) can vary depending on the characteristics of the sound. For
  \emph{complex} sounds without clear armonics (such as a plosive
  sound), shortened windows improve the performance. For \emph{simple}
  sounds (such as a music instrument), large windows are better.
\end{itemize}

\section{Channel coupling}
Most of the time, similar sounds are transported in the channels of a
non-mono audio signal. Channel coupling decreases inter-channel
redundancy, usually, using prediction techniques.

\section{Perceptual quantization \& white-noise filling}
Depending on the desired output bit-rate and the frequency (see the ATH
model), the SAM applies a different quantization step to barks (see
Section\textasciitilde{}\ref{sec:ATH}). Roughly speaking, the higher the
compression ratio, the larger the quantization step and therefore, the
quantization noise; and the higher the frequency, the wider the bark.
Notice also that the perception of a tone in a bark depends also on the
temporal masking.

At decoding time, those barks that suffered the biggest lossess are
usually filled with
\href{http://en.wikipedia.org/wiki/White_noise}{white noise} in order to
\href{http://simplynoise.com/}{increase the perceived
    quality}.

\subsection{MDCT (Modified Discrete Cosine Transform)}
\begin{itemize}
\item
  Equivalent to apply a
  \href{http://en.wikipedia.org/wiki/Filter_bank}{bank} of \(N\)
  filters.
\item
  Determines the correlation between a set of \(2N\) numbers (samples)
  and \(N\)
  \href{http://en.wikipedia.org/wiki/Orthogonality}{orthogonal}\footnote{Two
    functions/signals are orthogonal if it is impossible to obtain one
    of them by means of the other.}
  \href{http://guru.multimedia.cx/mdct/}{cosine functions}. Therefore,
  at the input of the DCT there are \(2N\) samples and at the output,
  \(N\) coefficients.
\item
  The MDCT coefficients \(S[w]\) of the PCM samples \(s[n]\) are defined
  as:
   \begin{equation}
    S[w] = \sum_{n=0}^{2N-1}s[n]cos\Big[\frac{\pi}{N}(n+\frac{1}{2}+\frac{N}{2})(w+\frac{1}{2})\Big].
    \label{eq:MDCT}
  \end{equation}
\end{itemize}

\section{Entropy coding}
VLC of MDCT coefficients.

\subsection{Stream format}
\svg{Ogg_Stream}{800}

\subsection{Window}
\svg{window}{600}
\svg{window_spectrum}{600}
\begin{comment}
  \title{\href{https://wiki.xiph.org/Vorbis-tools}{Vorbis tools}}

\maketitle
\tableofcontents


\chapter{oggenc}

\section{What is \href{http://freecode.com/projects/oggenc}{\texttt{oggenc}}?}
\begin{itemize}
\item OggEnc is a flexible and fully featured command line encoder for
the Ogg Vorbis format.
\end{itemize}

\section{Simplest encoding}
\begin{verbatim}
$ oggenc somefile.wav # Produces output as somefile.ogg
\end{verbatim}

\section{Specifying an output filename}
\begin{verbatim}
$ oggenc somefile.wav -o out.ogg
\end{verbatim}

\section{Specifying an encoding VBR (Variable Bit-Rate) of 256 Kbps}
\begin{verbatim}
$ oggenc infile.wav -b 256 out.ogg 
\end{verbatim}

\section{Specifying a maximum and average bitrate}
\begin{verbatim}
$ oggenc infile.wav --managed -b 128 -M 160 out.ogg 
\end{verbatim}

\section{Specifying a quality}
\begin{verbatim}
$ oggenc infile.wav -q 6 out.ogg 
\end{verbatim}

\section{Transcoding: Downsampling and downmixing to 11 kHz mono before encoding}
\begin{verbatim}
$ oggenc --resample 11025 --downmix infile.wav -q 1 out.ogg 
\end{verbatim}

\section{Adding some info about the track}
\begin{verbatim}
$ oggenc somefile.wav -t "Title" -a "Artist" -l "Album" -c "OTHERFIELD=Other comments"
\end{verbatim} 

\section{Encoding from stdin to stdout}
\begin{verbatim}
oggenc -
\end{verbatim}


\chapter{oggdec}

\chapter{ogg123}
A simple command line Ogg Vorbis decoder and player

\chapter{ogginfo}

\chapter{\texttt{vcut}}

\chapter{\texttt{vorbiscomment}}
\end{comment}
